\documentclass[12pt]{article}
\usepackage{natbib}
% \usepackage[hypertex]{hyperref}
\usepackage{bm}
\usepackage{amsfonts}
\usepackage{graphicx}


\oddsidemargin 0.0mm
\evensidemargin 0.0mm
\textwidth 160mm
\topmargin -10mm
\textheight 230mm
% \pagestyle{empty}

%%%%%%%%%%%%%%%%%%%%%%%%%%%%%%%%%%%%%%%%%%%%%%%%%%%%%%%%%%%%%%%%%%%

\newcommand{\threej}[6]
{ \left(\begin{array}{ccc}
#1&#2&#3\\
#4&#5&#6
\end{array}\right) }

\newcommand{\sixj}[6]
{ \left\{\begin{array}{ccc}
#1&#2&#3\\
#4&#5&#6
\end{array}\right\} }

\def\separation {0.5cm}
\def\ion#1#2  {#1\,{\small {#2}} }
\def\B{\textsc{BayesME}}

%%%%%%%%%%%%%%%%%%%%%%%%%%%%%%%%%%%%%%%%%%%%%%%%%%%%%%%%%%%%%%%%%%%


% For generation of the HTML manual with tth:
% \def\tthdump#1{#1}      % For generating TeX source; ignored by tth
% Redefine symbols problematic for the browser:
%%tth:\def\ga{\hbox{$>\sim$}}
%%tth:\def\la{\hbox{$<\sim$}}
%%tth:\def\Mo{\hbox{$M_o$}}
%%tth:\def\Lo{\hbox{$L_o$}}
%%tth:\def\Mdot{\hbox{$M^{dot}$}}
%%tth:\def\Ivezic{Ivezic}

%%tth:\begin{html}<TITLE>User Manual for MOLPOP-CEP</TITLE>\end{html}
%%tth: This HTML file was generated from the TeX source by
%%tth: the translator TTH, which uses symbol fonts.  These fonts are
%%tth: not normally enabled for Netscape running under X, because of
%%tth: the way Netscape groups its fonts. If your browser has problems
%%tth: displaying the math symbols in this manual, an easy fix can be found
%%tth: on the TTH website at
%%tth:\begin{html}<A HREF="http://hutchinson.belmont.ma.us/tth/Xfonts.html">http://hutchinson.belmont.ma.us/tth/Xfonts.html</A>\end{html}
%%tth:\begin{html}<HR>\end{html}
%%%%%%%%%%%%%%%%%%%%%%%%%%%%%%%%%%%%%%%%%%%%%%%%%%%%%%%%%%%%%%%%%%%

\begin{document}

\title                  {\sc User Manual for \B}

\author{ A. Asensio Ramos, M. J. Mart\'{\i}nez Gonz\'alez, J. A. Rubi\~no Mart\'{\i}n\\\\\\
         Instituto de Astrof\'{\i}sica de Canarias\\
         38205, La Laguna, Tenerife, Spain\\\\
        \\[0.5in] \today}
\date{}
\maketitle

\newpage

\tableofcontents

\newpage

\section*{Disclaimer}

This software is distributed ``as is'' and the authors do not take any responsability for
possible errors derived from its use by others. Apply it with care and
never trust the output without a careful meditation. \B\ can be freely used
provided that its origin is properly acknowledged and the reference Asensio Ramos, Mart\'{\i}nez
Gonz\'alez \& Rubi\~no Mart\'{\i}n (2007; A\&A, 476, 959) is cited and acknowledged in any
publication achieved with it. Before using \B\ we recommend the user to read carefully this
paper. Please, 
send us bug reports, comments and suggestions of possible improvements.
We point out that \B\ will be improved over the years, but it is now ready for a number of
interesting applications.

\section*{License}
\emph{This software is Copyright 2010 Andr\'es Asensio Ramos, Mar\'{\i}a Jes\'us
Mart\'{\i}nez Gonz\'alez \& Jose Alberto Rubi\~no Mart\'{\i}n. The software
may be used for experimental and research purposes only. Commercial use is
not permitted without prior agreement of the Copyright holders. The
software may be shared or distributed under the same restrictions, provided
all such users are made aware of this agreement. The software may not be
sold in whole or within part of a larger software product, whether in
source or binary forms.}

\newpage

%%%%%%%%%%%%%%%%%%%%%%%%%%%%%%%%%%%%%%%%%%%%%%%%%%
%%%%%%%%%%%%%%%%%%%%%%%%%%%%%%%%%%%%%%%%%%%%%%%%%%
\section{Introduction}

\B\ is a computer program for doing Bayesian inference on the
parameters of the Milne-Eddington model for the interpretation of
Stokes profiles. The standard version is appropriate for spectro-polarimetric
data. There are additional versions for doing inference with a 
microstructured Milne-Eddington atmosphere and another one for dealing
with filter-polarimeter data. The code is written in standard Fortran 90, with small IDL
scripts for analyzing the output. The code relies on some LAPACK routines that
are supplied in the \texttt{LAPACK} directory and on the Multinest v2.7 sampling
method supplied in the \texttt{NESTED} directory\footnote{http://www.mrao.cam.ac.uk/software/multinest/}. 

%%%%%%%%%%%%%%%%%%%%%%%%%%%%%%%%%%%%%%%%%%%%%%%%%%
%%%%%%%%%%%%%%%%%%%%%%%%%%%%%%%%%%%%%%%%%%%%%%%%%%
\section{Uncompressing and compiling \B}

The package comes in a single compressed file \texttt{bayesme\_ddmmmyy.tar.gz}, where
the version is indicated with the date of the package. After
unpacking with 
\begin{verbatim}
tar zxvf bayesme_ddmmmyy.tar.gz, 
\end{verbatim}
the \B\ directory
will contain the following subdirectories:

\begin{enumerate}
\item
{\tt Source} contains the Fortran 90 sources and a makefile that can be used
to build the binary file.
\item
{\tt Source\_filter} contains the Fortran 90 sources of the filter-polarimeter version and a makefile that can be used
to build the binary file.
\item
{\tt Source\_misma} contains the Fortran 90 sources of the microstructured atmosphere version and a makefile that can be used
to build the binary file.
\item
{\tt OBSERVATIONS} contains the observations and the definition of the filter
positions for the filter-polarimetry version.
\item
{\tt MARKOVCHAINS} contains the results of the sampling process for doing
Bayesian inference.
\item
{\tt MANUAL} contains this manual. 
\end{enumerate}

The code has been tested on Linux platforms using the Intel Fortran
Compiler (\texttt{ifort}). The compilation 
of the F90 code (in directory \texttt{FORTRAN}) is performed
with the supplied \texttt{makefile}. It is quite simple and easy to modify, and
contains additional comments about compiling. The
default compiler is the \texttt{ifort}, although you can try to use any other
compiler through the variable \texttt{COMPILER}. In order to obtain the executable file, just type:
\begin{verbatim}
       make all
\end{verbatim}
in the directory \texttt{FORTRAN} of \B. After compiling and linking, the executable is copied to the root directory of
\B.

The generated object and module files can be cleaned typing:
\begin{verbatim}
       make clean
\end{verbatim}

%%%%%%%%%%%%%%%%%%%%%%%%%%%%%%%%%%%%%%%%%%%%%%%%%%
%%%%%%%%%%%%%%%%%%%%%%%%%%%%%%%%%%%%%%%%%%%%%%%%%%
\section{Configuration files}
\subsection{Standard \B}
Here we describe the input configuration file line by line. The example
is the file \texttt{config}.

\begin{verbatim}
# Sample magnetic field uniformly in vector (B and theta)
0 0
\end{verbatim}
These two numbers are set to 0 if you want to sample the magnetic field
strength and the inclination uniformly. In other words, these define the
priors over these variables. It is customary to set them to 0.

\begin{verbatim}
# Calculate posterior for combinations of parameters
1
\end{verbatim}
Once the sampling has been carried out, if you want to calculate
the posterior for a derived quantity, these lines indicate which
combination. For instance, if you want to calculate the magnetic
flux density.

\begin{verbatim}
# Write the number of variables, the variables and the function
3 
\end{verbatim}
Number of model parameters on which the derived quantity depends on.

\begin{verbatim}
'alpha1' 'B1' 'theta1'
'alpha1*B1*cos(theta1*3.1415927/180.0)'
\end{verbatim}
You define the parameters in the first line separated by spaces
and the derived quantity in the second line. This formula will
be parsed by the code.

\begin{verbatim}
# --------- Observations and lines -------------
# File with the observations
'test.per'
\end{verbatim}
File with the observations. This file consists of a first line with the number of
wavelength points. Then, each line consists of: wavelength in \AA\, the Stokes
parameters $I/I_c$, $Q/I_c$, $U/I_c$, $V/I_c$, and the standard deviation of the
noise associated with each Stokes parameter and at each wavelength. Examples of
these files can be found in the directory \texttt{OBSERVATIONS}.

\begin{verbatim}
# File with the stray-light contamination (if applied)
'stray_light.per'
\end{verbatim}
File that contains the stray-light profile if used. This file is exactly like an
observation but with the four columns for the noise missing.

\begin{verbatim}
# File with the output chain
'MARKOVCHAINS/test'
\end{verbatim}
All the output will start with this name and some extensions will be added to
indicate different outputs.

\begin{verbatim}
# Stokes parameters' Weights
1d0 1d0 1d0 1d0
\end{verbatim}
If you want to weight a Stokes parameter more than another, do it here. However,
this is not the Bayesian way to proceed since this is equivalent to increasing
or decreasing the noise associated with the Stokes parameter and you have
to find a justification for that.

\begin{verbatim}
# Number of lines
2
\end{verbatim}
Number of spectral lines to consider.

\begin{verbatim}
# Atomic parameters file (maximum 10 characters)        
lines
\end{verbatim}
File with the atomic information. The file is self-explaining.

\begin{verbatim}
# Which line to synthesize (ZE/HF and the index. Repeat for each line)
ZE
0
ZE
1
\end{verbatim}
Lines to include from the file. The first string indicates if the line is
treated in the Zeeman regime or including hyperfine structure. At the moment,
only the Zeeman option is available. The second line indicates the index
of the line in the file indicated in the line before.

\begin{verbatim}
# mu angle
1.d0
\end{verbatim}
Cosine of the heliocentrinc angle.

\begin{verbatim}
# Number of components
2
\end{verbatim}
Number of Milne-Eddington components to use.

\begin{verbatim}

# ---------- Components ------------------------
# Component 1 - INDEX, INVERT (0/1), INITIAL VALUE, MINIMUM, MAXIMUM

# Magnetic field (Gauss)         
1 1000.0  0.d0  3000.d0
# Inclination 
1 20.0   0.d0  180.d0
# Azimuth
1 45.d0  0.d0  360.d0
# Doppler broadening (in Angstroms) 
1 0.02d0   0.01d0  0.08d0
# Macroscopic velocity (in km/s)
1 0.0d0  -5.0d0  5.d0
# Damping (in Angstroms)
1 0.0d0   0.d0   0.2d0
# Value of B1/Bo (source function gradient)  
1 4.0  0.d0 40.d0
# Line strength eta (one value per line)
1 8.5   0.0d0  40.0d0
1 8.5   0.d0   40.d0
# Filling factor (<0 if stray-light is used)
1 0.8d0 0.d0 1.d0

# ---------- Components ------------------------
# Component 1 - INDEX, INVERT (0/1), INITIAL VALUE, MINIMUM, MAXIMUM

# Magnetic field (Gauss)         
0 0.0  0.d0  3000.d0
# Inclination 
0 30.0   0.d0  180.d0
# Azimuth
0 45.d0  0.d0  180.d0
# Doppler broadening (in Angstroms) 
0 0.03d0   0.01d0  0.08d0
# Macroscopic velocity (in km/s)
1 0.0d0  -5.0d0  5.d0
# Damping (in Angstroms)
0 0.0d0   0.d0   0.1d0
# Value of B1/Bo (source function gradient)  
0 2.0  0.d0 40.d0
# Line strength eta (one value per line)
0 10.5   0.0d0  40.d0
0 10.5   0.0d0  40.d0
# Filling factor (<0 if stray-light is used)
1 -0.2d0 0.d0 1.d0
\end{verbatim}
These lines describe the model atmospheres. In this case, two components are used.
Each line presents the following information: i) 0/1 to indicate if the parameter
is kept fixed or you want to infer its value, ii) value of the parameter, which is
only used if it is fixed, iii) two numbers indicating the range of variation of
the parameter. Uniform priors are assumed for all parameters truncated to the
range provided in these lines. Make sure that the value provided for all parameters
is inside the range of variation. If not, the code might give strange results.
If you want to infer the filling factor of the components, remember to use
values that add up to 1. If you want the last component to be a stray-light
contamination read from a file, use a negative number like in the example (you can
also infer the Doppler velocity of this component). If you want a normal
Milne-Eddington component, use a positive number.

\subsection{\B\ for filter-polarimeters}
The configuration file is now \texttt{config\_filter}. There is
only a difference with the previous configuration file. After configuring
the file with the observations, you will find the following lines:

\begin{verbatim}
# File with the definition of the filters
'OBSERVATIONS/FILTERS/imax.filters'
\end{verbatim}
This indicates the file that contains the filters to be applied for synthesizing
the observations. The file contains: i) two lines of comments, ii) the number of
filters and the number of wavelengths on which they are defined, iii) another comment
line, iv) the positions in wavelength of the filters (central wavelength if you want),
v) another comment and vi) lines with the wavelength and the transmission of each
filter normalized to unit area.

\subsection{\B\ for Milne-Eddington microstructured atmospheres}
This option is still under development and has not been fully tested.
Use it at your own risk. The configuration file is now \texttt{config\_misma}. The
differences come now on the definition of the model parameters. There is an
additional line at the beginning of each component, indicating if the
component is optically thick or optically thin:

\begin{verbatim}
# Optically thin (index of component to which it is associated) or not (0)
0
\end{verbatim}
If the value is 0, the definition of the atmosphere is exactly like in the
standard code. If it is $n>0$, the component is optically thin and associated
to component $n$ defined before. If this is the case, this component shares
the thermodynamical parameters with component $n$ and the definition only needs

\begin{verbatim}
# Magnetic field (Gauss)
1 0.0  0.d0  3000.d0
# Inclination
1 30.0   0.d0  180.d0
# Azimuth
1 45.d0  0.d0  180.d0
# Doppler broadening (in Angstroms)
1 0.03d0   0.01d0  0.08d0
# Macroscopic velocity (in km/s)
1 0.0d0  -5.0d0  5.d0
# Damping (in Angstroms)
0 0.0d0   0.d0   0.1d0
# Filling factor (<0 if stray-light is used)
1 0.2d0 0.d0 1.d0
\end{verbatim}
The filling factor of this component is the internal one with component $n$.

\section{Running the code}
The code is run with the following lines, depending on the code you want
to use:
\begin{verbatim}
./bayesme config
./bayesme_filter config_filter
./bayesme_misma config_misma
\end{verbatim}

\section{Output}
The output is organized in several files, some of them are of
interest. Others are just temporal files or non-interesting files
for you. For instance if the \texttt{test} root is used,
we will find:
\begin{itemize}
\item \texttt{test.params\_inverted}. A list of the parameters over which inference has been carried out.
\item \texttt{test.stokes\_map}. The Stokes parameters synthesized at the value of the parameters that
maximize the posterior distribution. This is the so-called maximum a-posteriori (MAP) solution and
should be equivalent to the least-squares solution obtained with standard codes.
\item \texttt{testpost\_equal\_weights.dat}. This file contains samples of the posterior for all
parameters, plus the likelihood associated with it. Almost all information present in the
rest of files can be extracted from this samples.
\item \texttt{test.hist1D}. This file contains the marginal posteriors for the parameters, plus
those of the derived quantities. The first line indicates the number of histograms of
model parameters and the number of histograms of derived quantities. For each posterior,
you will find an index and the number of bins. Then, the following three columns contain
the bins and the histograms, with the second being a Gaussian-smoothed version of the 
third column, which is just the plain histogram.
\item \texttt{test.confidence}. This file contains, for each parameter, the median of each marginal 
posterior, together with 68\% and 95\% confidence intervals. Finally, the MAP value of
the parameter is also shown.
\item \texttt{test.percentiles}. Different percentiles of each marginal posterior.
\item \texttt{teststats.dat}. This file contains an estimation of the evidence and the
total number of high-probability regions found, together with some information
about each one.
\end{itemize}

\section*{Acknowledgements}
Finantial support by
the Spanish Ministry of Education and Science through projects AYA2007-63881, AYA2007-67965-C03-01 and
the European Commission through the SOLAIRE network (MTRN-CT-2006-035484) are gratefully acknowledged. 

% \bibliographystyle{apj}
% \bibliography{/home/aasensio/Dropbox/biblio}

\end{document}
